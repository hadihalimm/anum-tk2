%-----------------------------------------------------------------------------%
\chapter{\babTiga}
\label{bab:3}
%-----------------------------------------------------------------------------%
Setelah berhasil menyusun rumus, tim penulis kemudian menulis implementasi kode menggunakan
bahasa MatLab/Octave. Tim penulis menulis memisahkan kode menjadi dua bagian berdasarkan
tujuan dari kode tersebut.


% %-----------------------------------------------------------------------------%
\section{Memecahkan \f{Natural Cubic Spline} secara Numerik}
\label{sec:numericCubicSpline}
% %-----------------------------------------------------------------------------%
Dalam menulis kode, penulis sepenuhnya menggunakan rumus-rumus yang telah dijabarkan pada
\bab~\ref{bab:2}. Pada awal perhitungan, perlu dicari $\delta_i$ dan $\Delta_i$
sesuai dengan yang dijabarkan pada \equ~(TODO) dan \equ~(TODO). Agar menghindari
ambiguitas, $\delta_i$ akan ditulis sebagai \code{h} pada implementasi.

Karena kedua rumus yang dijabarkan penuh dengan deret, maka tim penulis
memanfaatkan sifat dari data struktur vektor dalam Octave untuk memudahkan
implementasi. Tim penulis memanfaatkan struktur data tersebut dengan membangun
\f{slice} dari vektor transpos menggunakan input $x$ dan $y$. \lst~\ref{code:h_i}
merupakan contoh kode yang memanfaatkan sifat struktur data vektor di Octave.
Teknik yang sama juga digunakan untuk menghitung $\Delta_i$.

\lstinputlisting[
	language=Octave,
	caption=Sampel kode untuk menghitung $\delta_i$ secara numerik,
	label=code:h_i
]{assets/codes/h_i.m}

Setelah $\delta_i$ dan $\Delta_i$ berhasil didapatkan, selanjutnya tim penulis
menerapkan \equ~(TODO) penggunakan operator \code{./} (pembagian setiap elemen)
untuk membagi setiap elemen $\Delta_i$ dengan $\delta_i$. Hasil dari pembagian
setiap elemen $\Delta_i$ dengan $\delta_i$ akan disimpan pada variabel \code{t},
seperti pada \lst~\ref{code:t_i}. Hasil yang disimpan akan digunakan untuk
membangun $b$ sistem persamaan linier $A x = b$.

\lstinputlisting[
	language=Octave,
	caption=Sampel kode untuk menghitung \equ~(TODO) secara numerik,
	label=code:t_i
]{assets/codes/findT.m}

Berdasarkan \equ~(TODO), kita dapat menggunakan teknik yang sama pada
\lst~\ref{code:h_i} untuk menyederhanakan kode implementasi. Dari $n$ titik,
diperoleh $n-2$ persamaan. Agar sistem persamaan linier tidak \f{underdetermined}
dan sifat natural dari \f{Cubic Spline} terjaga, maka ujung atas dan bawah vektor
yang dihasilkan ditambahkan $0$. Sehingga implementasi bagian $b$ pada sistem
persamaan $A x = b$ dapat ditulis seperti \lst~\ref{code:baxb}.

\lstinputlisting[
	language=Octave,
	caption={Sampel kode untuk menyusun bagian $b$ dari $A x = b$},
	label=code:baxb
]{assets/codes/buildB.m}

Untuk menyusun bagian $A$ dari sistem persamaan $A x = b$, tim penulis perlu
menyusun matriks tridiagonal seperti pada \equ~(TODO). Implementasi dapat
dilihat pada \lst~\ref{code:Aaxb}.

\lstinputlisting[
	language=Octave,
	caption={Sampel kode untuk menyusun bagian $A$ dari $A x = b$},
	label=code:Aaxb
]{assets/codes/buildA.m}

Setelah mendapat bagian $A$ dan $b$ pada sistem linier tridiagonal $A x = b$,
tim penulis menyelesaikan $x$ dengan fungsi eliminasi Gauss di Octave. Hasil
penyelesaian $x$ merupakan koefisien $c$ pada fungsi interpolan $S_i$.
Selanjutnya penulis mencari koefisien $b$ dan $d$ seperti yang dijabarkan
pada \equ~(TODO).

Karena seluruh koefisien sudah didapat, maka secara teknis kita dapat menyusun
fungsi-fungsi interpolan $S_i$. Namun, sintaks Octave tidak memungkinkan untuk
sebuah fungsi menghasilkan lebih dari 1 fungsi. Limitasi ini membuat tim
penulis untuk langsung mengevaluasi setiap titik pada $x_0$ sampai $x_{n-1}$.
Pada \lst~\ref{code:eNatureCubic} dapat terlihat algoritma untuk menggunakan
$S_i$ untuk mengevaluasi titik yang berada diantara $x_(i-1)$ dan $x_i$.

\lstinputlisting[
	language=Octave,
	caption={Sampel kode untuk evaluasi \f{Natural Cubic Spline} pada titik-titik $x$},
	label=code:eNatureCubic
]{assets/codes/evaluateNaturalCubic.m}

Keseluruhan kode akhir yang lengkap tersedia pada \apdx~\ref{appendix:naturalCubicSpline}.

% %-----------------------------------------------------------------------------%
\section{\f{Cubic Spline} Parametrik}
\label{sec:parametricSpline}
% %-----------------------------------------------------------------------------%
Pada \sect~\ref{sec:numericCubicSpline} telah dijelaskan bagaimana cara menyusun
fungsi \f{natural cubic spline} secara numerik. Agar dapat menghasilkan
\f{natural cubic spline} yang bersifat parametrik, tim penulis membungkus fungsi
\f{natural cubic spline} yang didapat pada \sect~\ref{sec:numericCubicSpline} ke
dalam fungsi bernama \f{parametric cubic spline}. Fungsi tersebut mengevaluasi
sembarang titik terhadap titik $x$ dan $y$ secara terpisah.

\lstinputlisting[
	language=Octave,
	caption={Sampel kode \f{natural cubic spline} secara parametrik.},
	label=code:parametricCubic
]{assets/codes/parametricSpline.m}

Dapat terlihat pada \lst~\ref{code:parametricCubic} bahwa untuk setiap nilai $x$
dan $y$ yang terpisah, keduanya akan mempunyai fungsi interpolan yang berbeda.
Hasil dari interpolasi \f{cubic spline} secara parametrik akan disimpan pada
variabel $p$ dan $q$. Kedua nilai tersebut dapat langsung diplot menggunakan
fungsi bawaan Octave. Kode akhir dari \f{cubic spline} parametrik dapat dilihat
pada \apdx~\ref{code:appendix:parametricCubic}.